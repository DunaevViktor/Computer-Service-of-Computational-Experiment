\documentclass[a4paper, 14pt]{extreport} %размер бумаги устанавливаем А4, шрифт 12пунктов
\usepackage{cmap} % для кодировки шрифтов в pdf
\usepackage[T2A]{fontenc} % кодировка с переносами
\usepackage[utf8]{inputenc} %кодировка шрифта
\usepackage[english,russian]{babel} %многоязычность
\usepackage{indentfirst} % отделять первую строку раздела абзацным отступом тоже


\usepackage{ulem} % подчеркивания
\linespread{1.1} %полуторный интервал

\renewcommand{\rmdefault}{ftm} % Times New Roman
\usepackage{pscyr}
\frenchspacing

\usepackage[]{graphicx}
\graphicspath{{Pictures/}}
\usepackage[usenames,dvipsnames]{color} % названия цветов
\usepackage{amssymb,amsfonts,amsmath,amsthm,mathtext,cite,enumerate,float} %подключаем нужные пакеты расширений

\usepackage[font=small,labelfont=bf]{caption} 


% \usepackage{makecell}
\usepackage{multirow} % улучшенное форматирование таблиц

\usepackage[unicode, pdftex,pdfpagelabels,bookmarks,hyperindex,hyperfigures]{hyperref} % подключаем hyperref (для ссылок внутри  pdf)

% поля
\usepackage{geometry} % Меняем поля страницы
\geometry{left=3cm}% левое поле
\geometry{right=1cm}% правое поле
\geometry{top=2cm}% верхнее поле
\geometry{bottom=2cm}% нижнее поле
%======================================================================================================
% оглавление
\usepackage{tocloft}
\renewcommand{\cfttoctitlefont}{\hspace{0.38\textwidth} \bfseries\MakeUppercase}
\renewcommand{\cftbeforetoctitleskip}{-1em}
\renewcommand{\cftaftertoctitle}{\mbox{}\hfill \\ \mbox{}\hfill{\footnotesize }\vspace{-2.5em}}
\renewcommand{\cftchapfont}{\normalsize\bfseries \MakeUppercase{\chaptername} }
\renewcommand{\cftsecfont}{\hspace{31pt}}
\renewcommand{\cftsubsecfont}{\hspace{11pt}}
\renewcommand{\cftbeforechapskip}{1em}
\renewcommand{\cftparskip}{-1mm}
\renewcommand{\cftdotsep}{1}
\setcounter{tocdepth}{2}
%====================================================================================
%стиль заголовков для введения, заключения и т.д.
\newcommand{\empline}{\mbox{}\newline}
\newcommand{\likechapterheading}[1]{
	\begin{center}\textbf{\MakeUppercase{#1}}\end{center}
}
\makeatletter
\renewcommand{\@dotsep}{2}
\newcommand{\l@likechapter}[2]{{\@dottedtocline{0}{0pt}{0pt}{#1}{#2}}}
\makeatother
\newcommand{\likechapter}[1]{    
	\likechapterheading{#1}    
	\addcontentsline{toc}{likechapter}{\MakeUppercase{#1}}}

%===============================================================================
% стиль библиографии
\usepackage[square,numbers,sort&compress]{natbib}
\renewcommand{\bibnumfmt}[1]{#1.\hfill} % нумерация источников в самом списке — через точку
\renewcommand{\bibsection}{\chapter*{СПИСОК ИСПОЛЬЗОВАННЫХ ИСТОЧНИКОВ}} % заголовок специального раздела
%===============================================================================


%стили заголовков
\usepackage[bf]{titlesec}
\titleformat{\chapter}
{\large\filcenter}
{\bfseries {\MakeUppercase{\chaptertitlename} \thechapter.}}
{8pt}{\bfseries}

\titleformat{\section}
{\normalsize \bfseries}
{\thesection}
{1em}{}

\titleformat{\subsection}
{\normalsize}
{\thesubsection}
{1em}{}
\titlespacing*{\chapter}{0pt}{-30pt}{36pt}
\titlespacing*{\section}{\parindent}{*4}{36pt}
\titlespacing*{\subsection}{\parindent}{*4}{36pt}
%==============================================================

\usepackage{lscape} %альбомная ориентация страницы
\usepackage{tikz}
\usepackage{pgfplots}
\usepackage{pgf-pie}

\usepackage{titletoc}
\titlecontents{chapter}% <section-type>
[0pt]% <left>
{}% <above-code>
{\chaptername\ \thecontentslabel:\quad}% <numbered-entry-format>
{}% <numberless-entry-format>
{\titlerule*[0.5pc]{.}\contentspage}% <filler-page-format>

\usepackage[margin=10pt,font=small,labelfont=bf,labelsep=endash]{caption}

%\usepackage[font=small]{caption} 
\usepackage[tableposition=bottom]{caption} 
\usepackage[figureposition=top]{caption} 
\usepackage{subcaption} 
\DeclareCaptionFormat{newformat}{\fontsize{13pt}{13pt}\selectfont#1#2#3} 
\DeclareCaptionLabelFormat{value}{#2} 
\DeclareCaptionLabelSeparator{separator}{~---~} 
\captionsetup{singlelinecheck=false,labelsep=separator,format=newformat} 

\captionsetup[table]{justification=raggedright,labelformat=value,margin=0pt} 

\renewcommand{\thetable}{Таблица \arabic{table}}

\RequirePackage{caption} 
\DeclareCaptionLabelSeparator{defffis}{ — } 
\captionsetup{justification=centering,labelsep=defffis} 

\newcommand{\Table}[1]{ 
	\small {#1} 
}



\begin{document}
%	 \captiondelim{~---~} \addto\captionsrussian{ \def\figurename{Рисунок} }
\def\figurename{Рисунок}
\begin{titlepage}
	\linespread{1.1}
	\begin{center}
		\fontsize{14pt}{14pt}\selectfont
		\textbf{МИНИСТЕРСТВО ОБРАЗОВАНИЯ РЕСПУБЛИКИ БЕЛАРУСЬ}\\
		\textbf{БЕЛОРУССКИЙ ГОСУДАРСТВЕННЫЙ УНИВЕРСИТЕТ}\\
		\textbf{ФАКУЛЬТЕТ ПРИКЛАДНОЙ МАТЕМАТИКИ И ИНФОРМАТИКИ}\\
		\textbf{Кафедра компьютерных технологий и систем}\\
		\vspace{3.5cm}
		\fontsize{14pt}{14pt}\selectfont
		ДУНАЕВ ВИКТОР АНДРЕЕВИЧ\\
		\vspace{0.7cm}
		\textbf{ИСПОЛЬЗОВАНИЕ ДОПОЛНЕННОЙ РЕАЛЬНОСТИ ДЛЯ ПРОВЕДЕНИЯ ПРОСТЫХ ХИРУРГИЧЕСКИХ ОПЕРАЦИЙ}\\
		\vspace{0.7cm}
		\fontsize{14pt}{14pt}\selectfont
		Курсовая работа\\
	\end{center}
	\vspace{3cm}
	\fontsize{14pt}{14pt}\selectfont
	\hspace{-0.25cm}
	\def\arraystretch{1.2}
	\begin{tabular}{l@{\hspace{10.25cm}}l}
		& Научный руководитель:\\
		&доцент кафедры КТС ФПМИ,\\
		&кандидат физ.-мат. наук \\
		&Василевский Константин \\
		&Викторович\\
		\vspace{5.5cm} \\

	\end{tabular}
	\vspace{1cm}
	\begin{center}
		\fontsize{14pt}{14pt}\selectfont
		Минск, 2018
	\end{center}
\end{titlepage}
\newpage


\setcounter{page}{2}
\begin{center}
	\fontsize{14pt}{14pt}\selectfont
	\textbf{РЕФЕРАТ}\\
\end{center}

%\begin{flushleft}
	Курсовая работа, 28 с., 13 рис., 8 источников.
		
	Microsoft HoloLens, КАЛИБРУЕМЫЙ ОБЪЕКТ, Open CV, UNITY3D
		
	Объект исследования – калибруемый объект, представленный в виде изображения (набора изображений).
		
	Цели работы – исследовать и изложить методы создания приложения для использования дополненной и виртуальной реальности в проведении простых хирургических операций. В частности, обработка и калибровка изображения.
		
	Методы исследования – методы моделирования в среде Unity3D с помощью библиотеки Open CV, методы создания приложения виртуальной реальности в Android Studio 3.0.
	
	Результатом является приложение, обрабатывающее изображение.
		
	Полученные результаты могут быть использованы в учебных целях или в области медицины.
	
	\vspace{1cm}
%\end{flushleft}
\begin{center}
	\fontsize{14pt}{14pt}\selectfont
	\textbf{РЭФЕРАТ}\\
\end{center}

%\begin{flushleft}
	
	Курсавая праца, 28 с., 13 мал., 8 крыніц.
	
	Microsoft HoloLens, КАЛІБРАВАНЫ АБ’ЕКТ, Open CV, UNITY3D
	
	Аб’ект даследвання – калібраваны аб’ект, прадстаўленны ў выглядзе малюнка (набору малюнкаў).
	
	Мэта працы – даследваць і выкласці метады стварэння прыкладання для выкарыстання дапоўненнай і віртуальнай рэальнасці ў правядзенні простых хірургічных аперацый. У прыватнасці, апрацоўка і каліброўка малюнка.
	
	Метады даследвання – метады мадалявання ў асяроддзі Unity3D з дапамогай бібліятэкі Open CV, метады стварэння прыкладання віртуальнай рэальнасці ў Android Studio 3.0.
	
	Вынікам   з’яўляецца прыкладанне, апрацоўчае малюнак.
	
	Атрыманыя вынікі могуць быць выкарыстаны ў навучальных мэтах або ў галіне медыцыны.
	
	\vspace{0.5cm}
%\end{flushleft}
\begin{center}
	\fontsize{14pt}{14pt}\selectfont
	\textbf{SUMMARY}\\
\end{center}


%\begin{flushleft}

	Course work, 28 p., 13 pic., 8 sources.

	Microsoft HoloLens, CALIBRATED OBJECT, Open CV, UNITY3D
	
	The object of study – is a calibrated object represented as an image (set of image).
	
	The purpose of the work – is to research and present methods of creating applications for the use of augmented and virtual reality in simple surgical operations. In particular, image processing and calibration.
	
	Research methods – methods of modeling in the Unity3D environment using the Open CV library, methods of creating a virtual reality application in Android Studio 3.0.
	
	The result is an application that processes the image.
	
	The results can be used for educational purposes or in the field of medicine.
	
%\end{flushleft}
\newpage

\hypersetup{pdfborder={0 0 0}} %\graphicspath{{Pictures/}}
\renewcommand\contentsname{Оглавление} 
\tableofcontents

\chapter*{ВВЕДЕНИЕ}
\addcontentsline{toc}{chapter}{ВВЕДЕНИЕ}

Дополненная реальность (от англ. augmented reality, AR) — это технология представления контекстной информации и наложения ее в виде многослойных визуальных образов на объекты реального мира в режиме реального времени.
 
Дополненная реальность является основой принципиально нового интерфейса для обращения к информации и перехода взаимодействия с ней на новый интерактивный уровень. Отличие дополненной реальности от виртуальной заключается во взаимодействии компьютерных устройств с объектами реального мира.
 
Задача дополненной реальности — расширить информационное взаимодействие пользователя с окружением. Накладываемые посредством компьютерного устройства слои с контекстными объектами на изображение реальной среды носят вспомогательно-информативный характер.

Таким образом, информация, контекстно связанная с объектами реального мира, с помощью дополненной реальности, становится доступной пользователю в режиме реального времени.
 
Примеры применения дополненной реальности: отображение информации на лобовом стекле в современных истребителях и авто премиум класса, сетка золотого сечения и другие вспомогательные элементы на экране цифрового фотоаппарата, указатели траектории парковки автомобиля при помощи камеры заднего вида.

\newpage

\chapter{MICROSOFT HOLOLENS}
Microsoft HoloLens — очки смешанной реальности, разработанные Mic- rosoft. Используют 32-разрядную операционную систему Windows Hologra- phic (версия Windows 10).

В отличие от шлемов виртуальной реальности, которые передают картинку через непрозрачные светодиодные или жидкокристаллические дисплеи и стеклянные линзы, которые искажают картинку для правильного восприятия зрительным аппаратом человека, в устройстве Microsoft установлены прозрачные волноводные линзы. Это линзы с волнообразной призматической структурой, которые правильным образом преломляют и отправляют в глаз картинки с расположенных по бокам \cite{b1} микродисплеев. Картинка формируется на основе данных с датчиков местоположения шлема и окружающих объектов, и в результате пользователь видит перед собой «голограммы» — компьютерную графику, интегрированную в окружающую действительность, или смешанную реальность.

\vspace{0.5cm}

\section{Типы приложений и инструменты для их создания}
В оптике выходной зрачок — это виртуальная апертура оптической системы, и из системы могут выйти только лучи, проходящие через виртуальную апертуру (другими словами – это область с образом, видимая в окуляр). В данном случае, входной зрачок — глаз пользователя, а выходным является проекция. Для правильной работы системы (\ref{11}). необходимо, чтобы расширение выходного зрачка осуществлялось через максимальное расширение области, доступной обзору человеческого зрачка с любой его позиции. Этого позволяет добиться регулировка линз в вертикальной и горизонтальной плоскостях. 
\vspace{0.4cm}
\begin{equation}
\label{11}
\sigma_{rr}=\frac{2G}{1-2\mu}\left((1-\mu)\frac{\partial{U}}{\partial{r}}+2\mu\frac{U}{r}-(1+\mu)\alpha{T}\right)
\end{equation}
\begin{equation}
\label{12}
\sigma_{\theta\theta}=\frac{2G}{1-2\mu}\left(\mu\frac{\partial{U}}{\partial{r}}+\frac{U}{r}-(1+\mu)\alpha{T}\right)
\end{equation}
\vspace{0.4cm}

Для передачи картинки в HoloLens используются (\ref{12}) линзы с призматическими структурами — волноводами. Их трудно изготовить прямо в стекле, поэтому инженеры \cite{b8} покрывают линзы несколькими дифракционными решётками. Это нужно для того, чтобы «голограммы» отображались правильно, а пользователь не испытывал дискомфорт. Проходя через оптическую систему, изображение дифрагируется внутри волновода, отправляясь в точном направлении с определёнными цветами.

\section{Общая схема приложения}
Опуская все, что можно в этой жизни. Бессмысленный текст абзаца, как и наша тленная жизнь. Я мог бы цитировать Гёте, но я из Вилейки, а ты лишь печаль.
\vspace{0.6cm}

\subsection{Обработка входных данных.} 
Выполняя для (\ref{22}) преобразование Лапласа Рисунок \ref{ris11} и подставив решение (\ref{23}) в (\ref{21}) находим
\vspace{0.4cm}
\begin{equation}\begin{split}
\label{13}
\frac{1}{r^{3}}\int_{R}^{r}{x^{2}T(x,t)dx}=\frac{R^{3}}{r^{3}}\int_{R}^{r}{\left(\frac{x}{R}\right)^{2}T_{0}\left(\frac{R}{x}\right)erfc\frac{\frac{x}{R}-1}{\frac{2\sqrt{at}}{R}}d\left(\frac{x}{R}\right)}=\\
=\frac{T_0}{\xi^3}\int_{R}^{r}{\left(\frac{x}{R}\right)erfc\frac{\frac{x}{R}-1}{2\sqrt{\tau}}d\left(\frac{x}{R}\right)}=T_{0}\frac{1}{\xi^3}\int_{1}^{\xi}{yerfc\frac{y-1}{2\sqrt{\tau}}dy}=\frac{T_0}{\xi^3}F(\xi,\tau)
\end{split}\end{equation}
\vspace{0.4cm}

Обратимся к Рисунок \ref{ris12}

\begin{center}
	\begin{minipage}{0.7\linewidth}
		\includegraphics[width=\linewidth]{scrin}
		\captionof{figure}{Круговая диаграмма гендера}
		\label{ris11}
	\end{minipage}
\end{center}

\vspace{0.8cm}

\begin{center}
	\begin{minipage}{0.93\linewidth}
		\includegraphics[width=\linewidth]{screenshot}
		\captionof{figure}{Любимый стример вечерком}
		\label{ris12}
	\end{minipage}
\end{center}
\vspace{0.7cm}

\subsection{Переход из режима «чтение» в режим «калибровка».} 
Термоупругие напряжения в квазистатическом режиме определяются соотношениями Рисунок \ref{ris13}. В \ref{table} что-то там указано.
\begin{center}
	\begin{minipage}{0.9\linewidth}
		\includegraphics[width=\linewidth]{gist.png}
		\captionof{figure}{Диаграмма}
		\label{ris13}
	\end{minipage}
\end{center}

\chapter{ОСНОВЫ СТЕРЕОЗРЕНИЯ}
Для того чтобы построить графики зависимостей, описанные формулами (\ref{13}) зададим диапазон безразмерных \cite{b1} величин. Графики приведены на рисунках Графики построены с помощью программы MatLab. Далее на Рисунок \ref{figure:12pic} мы увидим нечто приятное.

\begin{center}
	\begin{minipage}{1\linewidth}
		\includegraphics[width=\linewidth]{moresc}
		\captionof{figure}{Составное изображение}
		\label{figure:12pic}
	\end{minipage}
\end{center}
\vspace{0.6cm}

\section{Проективная геометрия}
Этот интеграл можно вычислить.

\vspace{0.7cm} 
\begin{equation}
\label{21}
\frac{\partial^2{\left(\frac{T}{T_0}\right)}}{\partial{\left(\frac{r}{R}\right)^2}}+\frac{2\partial{\left(\frac{T}{T_0}\right)}}{\frac{r}{R}\partial{\left(\frac{r}{R}\right)}}=\frac{\partial{\left(\frac{T}{T_0}\right)}}{\partial{\left(\frac{at}{R^2}\right)}}
\end{equation}
\vspace{0.7cm} 

и после интегрирования окончательно получаем:

\vspace{2cm} 

\begin{equation}
\label{22}
\frac{\partial^2{\left(\frac{T}{T_0}\right)}}{\partial{\left(\frac{r}{R}\right)^2}}+\frac{2\partial{\left(\frac{T}{T_0}\right)}}{\frac{r}{R}\partial{\left(\frac{r}{R}\right)}}=\frac{\partial{\left(\frac{T}{T_0}\right)}}{\partial{\left(\frac{at}{R^2}\right)}}
\end{equation}
\vspace{0.3cm} 
\begin{equation}
\label{23}\begin{split}
T_0(0)=T_0=const,\\
U(r,0)=0
\end{split}\end{equation}

\vspace{0.6cm} 
\section{Модель проективной камеры}

%\bftext{\ref{tableable}: Какая-то таблица}

\begin{table}[h]
	\caption{Какая-то таблица}
	\centering
	\begin{tabular}{|l|l|l|l|l|}
		\cline{1-5}
		Длинное поле&Длинное поле&Длинное поле&Длинное поле&Поле\\
		\cline{1-5}
		Значение&Значение&Значение&Значение&Значение\\
		\cline{1-5}
	\end{tabular}
	\label{table}
\end{table}

Обратимся к биткоину ибо я его люблю Рисунок \ref{ris22}:
\begin{center}
	\begin{minipage}{1\linewidth}
		\includegraphics[width=\linewidth]{untitled2.png}
		\captionof{figure}{График биткоина}
		\label{ris22}
	\end{minipage}
\end{center}

\vspace{0.7cm} 
В этой части сошлемся на Рисунок \ref{ris23}. 
\begin{landscape} 
	\begin{center} 
		\begin{minipage}{0.9\linewidth} 
			\includegraphics[width=\linewidth]{scr.png} 
			\captionof{figure}{Альбомная картинка} 
			\label{ris23} 
		\end{minipage} 
	\end{center} 
\end{landscape} 
\chapter*{ЗАКЛЮЧЕНИЕ} 
\addcontentsline{toc}{chapter}{ЗАКЛЮЧЕНИЕ}

В работе были получены следующие основные результаты: исследованы и изложены основные методы создания приложения для использования дополненной и виртуальной реальности в проведении простых хирургических операций. В частности, обработка и калибровка изображения. Средой разработки была выбрана Unity3D, основной \cite{b8} библиотекой функций OpenCV. Так же с помощью Google VR SDK и Android Studio 3.0 было реализовано приложение виртуальной реальности, позволяющее понять и применить принципы стереозрения для обработки изображения. Полученные разработки могут быть в дальнейшем использованы другими разработчиками для усовершенствования и практического применения в ряде сфер, в частности в сфере медицины.
\newpage


\begin{thebibliography}{99}
	\addcontentsline{toc}{chapter}{СПИСОК ИСПОЛЬЗОВАННЫХ ИСТОЧНИКОВ}
	\bibitem{b1} Форсайт, Д. Компьютерное зрение. Современный подход / Д. Форсайт, Ж. Понс — М.: Издательство «Вильямс», 2004. — 928 с.
	\bibitem{b2} Baggio, D.L. OpenCV Computer Vision with Java / D.L. Baggio — Packt Publishing, 2015. — 174p.
	\bibitem{b3} Лисовицкий, А. Из чего состоит Microsoft HoloLens и как все это работает / А. Лисовицкий // Голографика. Отраслевое издание о бизнесе в области дополненной, смешанной и виртуальной реальности [Электронный ресурс]. —2016.—Режим доступа: https://holographica.space/articles/microsoft-hololens-10-6983  — Дата доступа:10.12.2017.
	\bibitem{b4} OpenCV documentation [Electronic resource] / OpenCV development team —2014.— Mode of access: https://docs.opencv.org/2.4/index.html  — Date of access: 27.11.2017
	\bibitem{b5} Bradski, G. Learning OpenCV. Computer Vision with the OpenCV Library / G. Bradski, A. Kaehler — O'Reilly Media, 2008. — 580p..
	\bibitem{b6} Hartley R. Multiple View Geometry in Computer Vision Second Edition / R. Hartley, A. Zisserman — Cambridge University Press, 2004. — 646p.
	\bibitem{b7} Getting started with VR View for Android [Electronic resource] / Google development team —2015.— Mode of access: https://codelabs.developers.google.com/  — Date of access: 19.05.2018
	\bibitem{b8} 360 Media [Electronic resource] / Google development team —2015.— Mode of access: https://developers.google.com/vr/discover/360-degree-media  — Date of access: 20.05.2018.
\end{thebibliography}
\end{document}